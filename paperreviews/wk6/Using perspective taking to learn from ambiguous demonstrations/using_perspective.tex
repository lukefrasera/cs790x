\documentclass{article}
\usepackage{hyperref}
\usepackage{natbib}
\usepackage[right=1in,top=1in,left=1in,bottom=1in]{geometry}

\begin{document}
\title{{\large Review} \\ Using perspective taking to learn from ambiguous demonstrations}
\author{Luke Fraser}
\date{\today}
\maketitle

% REFERENCE THE PAPER HERE ////////////////////////////////////////////////////////////////////
\begingroup
\renewcommand{\section}[2]{}
\bibliographystyle{plain}
\bibliography{references}
\endgroup

% /////////////////////////////////////////////////////////////////////////////////////////////
\section*{Summary}
% WRITE SUMMARY SECTION HERE //////////////////////////////////////////////////////////////////
In this paper the authors present a demonstration method that uses human perspective to manage ambiguities between the human teacher and the robot learner. It is important for robots to understand the perspective of the individual teaching them. In many cases the understanding of the teacher can differ from that of the robot do to perspective differences. This becomes obvious when the robot can see part of a scene that the teach cannot. This is an occlusion difference that would require the robot to understand the perspective of the human teacher to know that the teacher cannot see part of the scene.

To manage this the paper presents an architecture that takes into account both the robots perspective and the teachers perspective in the action the selection of the robot. The robot used for this experiment has two stereo eye pairs to understand its environment. One of the pairs is fixed behind the robot and the other is a top down view of the scene. The perspective of the teacher is taking by using a culling cone based on the teacher location and orientation in the scene. Then the robot can build a belief system based on what it believes the teacher can see.

The robot then uses the two different models of the environment to communicate to the teacher differences in the model by pointing. This will allow the teacher to see if the robot is illustrating an important distinction between the perspective of the robot and the teacher. This will allow ambiguities to be addressed and made clear. The robot then uses these beliefs to decide on the overall goal understanding. This is done through probability to determine the most likely desired outcome of the exercise.

A experiment is performed to validate the method. A teacher is instructed to teach a robot how to turn on several colored lights. The robot was successfully able to understand the difference in perspective for the simple case given.
% /////////////////////////////////////////////////////////////////////////////////////////////
\section*{Strengths}
% DISCUSS THE STRENGTHS OF THE PAPER //////////////////////////////////////////////////////////
The use of perspective to assist in flawed teaching is an excellent mechanism for getting a more accurate understanding of the teachers environment. This method is a much more human like way of learning from a teacher. People are very good at understanding the perspective of different people and use it often to make assumptions about teir environment. It would be interesting to see this method applied to a more complex task
% /////////////////////////////////////////////////////////////////////////////////////////////
\section*{Critique}
% DISCUSS THE CRITIQUE OF THE PAPER ///////////////////////////////////////////////////////////
The use of the environmentally located cameras make the perspective taking seem like a moot point. If the robot has such a complete view of its environment then the use of perspective taking is less important. This system would not function as well on a system that only had perception as a human does. The vision system should be located on the robot solely.  This makes the perspective problem more difficult.
% /////////////////////////////////////////////////////////////////////////////////////////////
\cite{chernova2010confidence}

\end{document}
