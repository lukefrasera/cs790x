\documentclass{article}
\usepackage{hyperref}
\usepackage{natbib}
\usepackage[right=1in,top=1in,left=1in,bottom=1in]{geometry}

\begin{document}
\title{{\large Review} \\ Confidence-based multi-robot learning from demonstration}
\author{Luke Fraser}
\date{\today}
\maketitle

% REFERENCE THE PAPER HERE ////////////////////////////////////////////////////////////////////
\begingroup
\renewcommand{\section}[2]{}
\bibliographystyle{plain}
\bibliography{references}
\endgroup

% /////////////////////////////////////////////////////////////////////////////////////////////
\section*{Summary}
% WRITE SUMMARY SECTION HERE //////////////////////////////////////////////////////////////////
In this paper the authors present a confidence based method for learning by demonstration. Learning by demonstration is a method in which robots can be trained to perform a task. A teacher demonstrates how to perform a given task while the robots watch. The robots then attempt to perform the task after watching a correct example of the task. Through these demonstration robots can learn much faster than alternative methods such as reinforcement learning. The method presented in this paper is called \emph{flex}ML\emph{f}D \emph{multirobot learning from demonstration}. \emph{flex}ML\emph{f}D is a demonstration learning method that allows a teacher to teach multiple robots at the same time collaboratively. A teacher performs demonstrations to a group of robots that are able to communicate with each other and share resources to learn.

The basis of the dmemonstration learning is from the Confidence Based Autonomy(CBA).

% /////////////////////////////////////////////////////////////////////////////////////////////
\section*{Strengths}
% DISCUSS THE STRENGTHS OF THE PAPER //////////////////////////////////////////////////////////

% /////////////////////////////////////////////////////////////////////////////////////////////
\section*{Critique}
% DISCUSS THE CRITIQUE OF THE PAPER ///////////////////////////////////////////////////////////
The time spent teaching a single robot vs seven robots is a linear conversion however it is cheaper to teach a single robot seven times over 7 robots once.
% /////////////////////////////////////////////////////////////////////////////////////////////
\cite{chernova2010confidence}

\end{document}
