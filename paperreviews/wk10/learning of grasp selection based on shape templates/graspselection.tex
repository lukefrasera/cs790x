\documentclass{article}
\usepackage{hyperref}
\usepackage{natbib}
\usepackage[right=1in,top=1in,left=1in,bottom=1in]{geometry}

\begin{document}
\title{{\large Review} \\ Learning of Grasp Selection Based on Shae-templates}
\author{Luke Fraser}
\date{\today}
\maketitle

% REFERENCE THE PAPER HERE ////////////////////////////////////////////////////////////////////
\begingroup
\renewcommand{\section}[2]{}
\bibliographystyle{plain}
\bibliography{references}
\endgroup

% /////////////////////////////////////////////////////////////////////////////////////////////
\section*{Summary}
% WRITE SUMMARY SECTION HERE //////////////////////////////////////////////////////////////////
In this paper the authors describe a method for choosing good grasping techniques for robotic arms. Grasping is a very important aspect of robotics. It is what enables a robot to interact with its environment. People are very good at understanding how to grab different objects in their environment and we can generalize this very well to new object that we come in contact with. With this ability we are almost unaware how often we perform these actions. We can conform to arbitrary objects quickly and smoothly interact with them without much thought. This is not yet possible with robots. robots are very slow at interacting with their environment.

Typically grasping in robotics is handled with very large database of a priori computed grasping schemes for a finite set of objects. This means that a robot must have prior knowledge of something that it wants to grasp. This is limiting and does not generalize well to arbitrary shapes or even shapes that are similar to known shape grasping schemes. To extend this work has been done to find similar shapes in the database with the current viewed different object. This however has proven to be even harder than storing larger datasets. It is difficult to rectify semantic information from sensor data. Applying to to infer a given object model s very difficult and hasn't produced good results in practice.

The idea is that robots can make wise decisions about grasping a particular object if it is similar to another object the robot knows how to grasp. In the paper they test the method on a PR2 and a Barrett WAM robot. As the robot attempts to grasp new objects it will learn and modify its knowledge to include successful grasps of the new object. Instead of building a object model they directly compare one objects sensor information to another to avoid the necessity of building an object model for arbitrary shapes.

The aurthors propose a local grasp shape descriptor which encodes hand sized grasp locations onto the object. These descriptors or templates can be matched at runtime to associate a given object to a proper grasping technique. As the algorithm runs it is able to include information from new grasps it learns so in the future interaction with object is fluid. Initial grasping locations are learned by demonstration. This allows a human to show the robot correct grasping locations of a set of objects. This builds a database for the robot to use for interacting with new arbitrary objects.
% /////////////////////////////////////////////////////////////////////////////////////////////
\section*{Strengths}
% DISCUSS THE STRENGTHS OF THE PAPER //////////////////////////////////////////////////////////
The method was successfully able to have a robot grasp arbitrary objects with a high success rate. The use of continued learning makes the approach improve as the robot grasps new objects. The robot should improve the more objects it interacts with.
% /////////////////////////////////////////////////////////////////////////////////////////////
\section*{Critique}
% DISCUSS THE CRITIQUE OF THE PAPER ///////////////////////////////////////////////////////////
The data base however will grow out of control with the life of the robot and eventually slow down in time. As well if the robot finds a comlpetely new object that it has not seen the algorithm will still fail to handle this. Even if the new object has known grasping locations the strange shape will not recognize with known shapes in the database.
% /////////////////////////////////////////////////////////////////////////////////////////////
\cite{goossens93}

\end{document}
