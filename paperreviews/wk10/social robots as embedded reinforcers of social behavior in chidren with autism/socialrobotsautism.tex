\documentclass{article}
\usepackage{hyperref}
\usepackage{natbib}
\usepackage[right=1in,top=1in,left=1in,bottom=1in]{geometry}

\begin{document}
\title{{\large Review} \\ Social Robots as Embedded Reinforcers of Social Behavior in Children with Autism}
\author{Luke Fraser}
\date{\today}
\maketitle

% REFERENCE THE PAPER HERE ////////////////////////////////////////////////////////////////////
\begingroup
\renewcommand{\section}[2]{}
\bibliographystyle{plain}
\bibliography{references}
\endgroup

% /////////////////////////////////////////////////////////////////////////////////////////////
\section*{Summary}
% WRITE SUMMARY SECTION HERE //////////////////////////////////////////////////////////////////
In this paper the authors present findings from a study on the effects of autistic children and interaction with robots. The study found that autistic children were more likely to children were more likely to interact verbally with a human robot pair than a human human pair or a human computer pair. The children interacted with the adult as often as they interacted with the robot.

The authors designed a randomized, controlled, crossover experiment to compare the effects of interactions with a social dinosaur robot against the effects of interactions with a human or an asocial novel technology~\cite{goossens93}. It was thought that the robot, adult, and touchscreen would be viwed as social and engaging, social and not engaging, and engaging and not social. Therefore it was hypothesized that the robot would produce the most interactions for the autistic child.

The participants varied in age from 4.6 to 12.8 years of age. The IQ of the participants was also verified to fall above 70. The participants had a mean IQ of 94.2 with a standard deviation of 11.7 and a min of 72. The robot was controlled using a Wizard of Oz Method. Meaning that the robot was controlled by a human operator with a predefined method of control to simulate robotic interactions robustly without the need for complex computer vision and other control methods.
% /////////////////////////////////////////////////////////////////////////////////////////////
\section*{Strengths}
% DISCUSS THE STRENGTHS OF THE PAPER //////////////////////////////////////////////////////////
The findings from the study show that there is a correlation between the use of the robot and the interaction levels of the child. This provides a strong motivation to further study the effects of the robots with the treatment of autistic children.
% /////////////////////////////////////////////////////////////////////////////////////////////
\section*{Critique}
% DISCUSS THE CRITIQUE OF THE PAPER ///////////////////////////////////////////////////////////
My concern with the use of robots and autistic children in the novelty effect. It is unclear with this study if the high level of interactions with the roboto were due to the novelty of the robot or the fact that the child was actually interested in the dinosaur and felt more comfortable with it than with people. Further testing would be necessary for this to be seen. A more long term study would be required.
% /////////////////////////////////////////////////////////////////////////////////////////////
\cite{goossens93}

\end{document}
