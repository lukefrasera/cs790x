\documentclass{article}
\usepackage{hyperref}
\usepackage{natbib}
\usepackage[right=1in,top=1in,left=1in,bottom=1in]{geometry}

\begin{document}
\title{{\large Review} \\ James Fahmous}
\author{Luke Fraser}
\date{\today}
\maketitle

% REFERENCE THE PAPER HERE ////////////////////////////////////////////////////////////////////
\begingroup
\renewcommand{\section}[2]{}
\bibliographystyle{plain}
\bibliography{references}
\endgroup

% /////////////////////////////////////////////////////////////////////////////////////////////
\section*{Summary}
% WRITE SUMMARY SECTION HERE //////////////////////////////////////////////////////////////////
James' talk was titled ``Data­Driven Scientific Discovery in the Big Data Era''. His focus was on using big data in a way to push scientific discovery. He believed that data could be used to help push scientific discovery further and faster by helping produce useful information that could be used by the scientific communities.

A central project that he worked on was with oceanic \emph{eddies}. An eddie is a vortex current in the ocean that spins in either a counter or clock wise rotation. These eddies form all over the oceans surface and come in a variation of sizes. Prior to his work there were no large scale, accurate data sets of eddy tracking on a global scale. Through the use of satellite imagery he was able to recognize and track eddies over many years. This formed a large dataset that could be used to perform much scientific analysis. The information was later used to understand the effect of eddies on hurricanes. The goal was to find conclusive evidence of the effect of the different eddy types on the strength of hurricanes.

% /////////////////////////////////////////////////////////////////////////////////////////////
\section*{Strengths}
% DISCUSS THE STRENGTHS OF THE PAPER //////////////////////////////////////////////////////////
James' talk was very well put together and organized in an interesting an compelling manner. He made you understand the importance of his research as well as understand the methods for understanding and producing useful information from the original untagged data set. He spoke in a very commanding way that expressed full understanding of his field. His talk was very nice and interesting.
% /////////////////////////////////////////////////////////////////////////////////////////////
\section*{Critique}
% DISCUSS THE CRITIQUE OF THE PAPER ///////////////////////////////////////////////////////////
Most of his talk was about generating Big-Data and less about categorizing it. His approach for finding eddies was not a big-data approach more of a vision problem. He only speaks of his analysis of the data at the end of his talk in a short manner. His method for handling the data was not a focus of the talk.
% /////////////////////////////////////////////////////////////////////////////////////////////
\cite{goossens93}

\end{document}
