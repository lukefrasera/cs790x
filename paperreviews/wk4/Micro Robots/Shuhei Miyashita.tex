\documentclass{article}
\usepackage{hyperref}
\usepackage{natbib}
\usepackage[right=1in,top=1in,left=1in,bottom=1in]{geometry}

\begin{document}
\title{{\large Review} \\ Shuhei Miyashita}
\author{Luke Fraser}
\date{\today}
\maketitle

% REFERENCE THE PAPER HERE ////////////////////////////////////////////////////////////////////
\begingroup
\renewcommand{\section}[2]{}
\bibliographystyle{plain}
\bibliography{references}
\endgroup

% /////////////////////////////////////////////////////////////////////////////////////////////
\section*{Summary}
% WRITE SUMMARY SECTION HERE //////////////////////////////////////////////////////////////////
Shuhei's talk was about micro-robots and self-assembling robots. The core of his presentation was on self-assembling robots meant for use in the human body. He took inspiration from origami to develop little robots that could self assemble. Robots were made from a 2D surface and then through activation from heat would bend into arbitrary shapes. This was accomplished by squashing a material that changes shape when heat is introduced with other rigid materials. He was successfully able to generate little robots that could be controlled by magnetic forces. The robots were able to complete simple tasks. Using the same concept of heat induced assembly Shuhei was also able to create a sulinoid that could provide some level of actuation.

The second part of his talk was about generating self assembly from pieces of a larger object. This section focused on designing a system with magnets similar to that of molecular synthesis using DNA. A simulation of magnets as proteins was generated and a test bed of self assembling pieces were able to react in an ever increasing manner. This showed that it was possible to create systems capable of forming larger systems from a magnetic system of components.

His work will hopefully benefit the medical industry, by providing alternative methods of fixing problems in the body through non-invasive means. The use of micro robots that could be ingested and then assembled into a larger more functional component would be useful for many aspect of the medical industry.
% /////////////////////////////////////////////////////////////////////////////////////////////
\section*{Strengths}
% DISCUSS THE STRENGTHS OF THE PAPER //////////////////////////////////////////////////////////
His talk was well formed and presented the material in a very understandable manner. He covered the description of his method well. He also provided well defined motivation for completing his research. Much of his research showed promising results that could be applied to make more complex robots capable of completing a more realistic goal.
% /////////////////////////////////////////////////////////////////////////////////////////////
\section*{Critique}
% DISCUSS THE CRITIQUE OF THE PAPER ///////////////////////////////////////////////////////////
There were certain times during his presentation that it would have been nice for him to go into slightly more detail. This would have improved some understanding that I may have missed.  
% /////////////////////////////////////////////////////////////////////////////////////////////
\cite{goossens93}

\end{document}
