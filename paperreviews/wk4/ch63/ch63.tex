\documentclass{article}
\usepackage{hyperref}
\usepackage{natbib}
\usepackage[right=1in,top=1in,left=1in,bottom=1in]{geometry}

\begin{document}
\title{{\large Review} \\ CH63 - Perceptual Robotics}
\author{Luke Fraser}
\date{\today}
\maketitle

% REFERENCE THE PAPER HERE ////////////////////////////////////////////////////////////////////
\begingroup
\renewcommand{\section}[2]{}
\bibliographystyle{plain}
\bibliography{references}
\endgroup

% /////////////////////////////////////////////////////////////////////////////////////////////
\section*{Summary}
% WRITE SUMMARY SECTION HERE //////////////////////////////////////////////////////////////////
In this chapter the authors provide an overview of perception in robotics. The chapter focuses on visual perception in robotics, specifically object recognition, movement representation, and synthesis models. Perception is an important aspect of robotics. Through perception robots are capable of recognizing many facets of human actions. For example if a robot were to need to react in a socially acceptable manner then the robot would need to understand and recognize human emotions and state. In order to recognize emotions robotic perception is required. If a robot is required to understand what objects are in its environment then visual perception is required to perform object recognition. Perception is an integral component of robotics that must be addressed.


% /////////////////////////////////////////////////////////////////////////////////////////////
\section*{Strengths}
% DISCUSS THE STRENGTHS OF THE PAPER //////////////////////////////////////////////////////////

% /////////////////////////////////////////////////////////////////////////////////////////////
\section*{Critique}
% DISCUSS THE CRITIQUE OF THE PAPER ///////////////////////////////////////////////////////////

% /////////////////////////////////////////////////////////////////////////////////////////////
\cite{chapter}

\end{document}
