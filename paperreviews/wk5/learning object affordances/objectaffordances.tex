\documentclass{article}
\usepackage{hyperref}
\usepackage{natbib}
\usepackage[right=1in,top=1in,left=1in,bottom=1in]{geometry}

\begin{document}
\title{{\large Review} \\ Learning Object Affordances: From Sensory--Motor Coordination to Imitation}
\author{Luke Fraser}
\date{\today}
\maketitle

% REFERENCE THE PAPER HERE ////////////////////////////////////////////////////////////////////
\begingroup
\renewcommand{\section}[2]{}
\bibliographystyle{plain}
\bibliography{references}
\endgroup

% /////////////////////////////////////////////////////////////////////////////////////////////
\section*{Summary}
% WRITE SUMMARY SECTION HERE //////////////////////////////////////////////////////////////////
I this paper the authors describe a method for robots to learn and understand their surroundings through experimentation as well as demonstration. The idea being that similar to children a robot should be able to test out different actions and understand the reaction. This allows robots to learn about their surroundings through interactions.

The robot had three defined actions: \emph{grasp}, \emph{tap}, and \emph{touch}. These actions can be used in any order to achieve a given task goal. The robot also interacted with a very limited set of objects. Different colored balls and cubes were used for the robot interactions. The robots leaning phase was broken down into sensory-motor coordination, world interaction, and imitation. The robot learns from the ground up what it can do and how to interact with its environment. After the robot learns about its environment the robot is then task with imitating a person to show that the robot has an understanding of its environment and abilities.

A Bayesian network was used to learn and train against the environment of the a priori action set and object set. It was trained using a Markov chain Monte Carlo method. The resulting system was able to learn about its environment and then use that understanding to imitate a humans simple actions on the object set.
% /////////////////////////////////////////////////////////////////////////////////////////////
\section*{Strengths}
% DISCUSS THE STRENGTHS OF THE PAPER //////////////////////////////////////////////////////////
The main contribution of this paper is the architecture development to produce a system that is capable of Independently learning about its environment. The robot was successfully able to mimic others interactions with objects after understanding what was possible through experimental actions with the objects.
% /////////////////////////////////////////////////////////////////////////////////////////////
\section*{Critique}
% DISCUSS THE CRITIQUE OF THE PAPER ///////////////////////////////////////////////////////////
The main issue with the method described in the paper is the scalability and robustness to complex interactions with its environment. The computation for more complex actions would increase the state space of known actions and objects making it harder to represent. As the number of objects and actions increase the state space will increase exponentially. As well the number of actions that have to be learned increase learning phase will grow far too fast for the system to understand its environemtn in a reasonable amount of time.
% /////////////////////////////////////////////////////////////////////////////////////////////
\cite{4456755}

\end{document}
