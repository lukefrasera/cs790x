\documentclass{article}
\usepackage{hyperref}
\usepackage{natbib}
\usepackage[right=1in,top=1in,left=1in,bottom=1in]{geometry}

\begin{document}
\title{{\large Review} \\ Emulating Self-reconfigurable Robots - Design of the SMORES System}
\author{Luke Fraser}
\date{\today}
\maketitle

% REFERENCE THE PAPER HERE ////////////////////////////////////////////////////////////////////
\begingroup
\renewcommand{\section}[2]{}
\bibliographystyle{plain}
\bibliography{references}
\endgroup

% /////////////////////////////////////////////////////////////////////////////////////////////
\section*{Summary}
% WRITE SUMMARY SECTION HERE //////////////////////////////////////////////////////////////////
In this paper the authors discuss their work on a reconfigurable robot that is able to change shape to represent many different robots with different capabilities. The hope is to produce a modular robot that is capable of being universal robot capable of transforming into a new robot to deal with challenges the current configuration can't handle. A lot of previous work has been used to produce different styles of walking from a base reconfigurable robot. Reconfigurable robots have been able to walk with different base structures such as two, three, and six legged walking as well as snake walking and etc. There are three different types of reconfigurable systems: chain, lattice, and mobile. Each type has different pros and cons associated with different configuration abilities.

The SMORES system presented in this paper was designed to be a more universal system that takes strategies from all the other types of reconfigurable systems. With this in mind the SMORE system is able to act as a chain, lattice, or mobile system. SMORES has 4DOF per robotic unit. Each unit is able to dock with another peer using permanent magnet docking ports. As well eac unit is capable of moving on its own with 2 of the docking ports acting as drive wheels. When connection is made the docks reach out and connect with another dock to make a strong connection. Once connected the docks can still rotate and move to allow different configuration and control. This is similar to a chaining system.

The SMORES system was successful at connecting and moving into different configurations. Although there are no external sensing capabilities the SMORES robots are able to configure themselves into many different orientations with their robust docking ports.
% /////////////////////////////////////////////////////////////////////////////////////////////
\section*{Strengths}
% DISCUSS THE STRENGTHS OF THE PAPER //////////////////////////////////////////////////////////
The ability of each of the robots to move on their own is a major advantage to many of the other systems. The simple design and limited moving parts make the SMORES robot ideal from a contorl perspective. This makes the robot more robust to failure and easier to program.
% /////////////////////////////////////////////////////////////////////////////////////////////
\section*{Critique}
% DISCUSS THE CRITIQUE OF THE PAPER ///////////////////////////////////////////////////////////
The lack of external sensing capabilities makes the fact that the robots can move standalone almost a moot point. Without external sensing the robots will become lost as soon as connection to a fellow robot is lost. Without an external tracking system maintaining accurate location of each robot mobility isn't as useful. The results of the paper are shown for only 2 robots. I assume this is due to the fact that they only built two of them. It would have been nice to see more of the SMORES bots interacting together to produce some of the configurations of prior work. This would show that the SMORES design is more universal in that it can take on all configuration of other chain, lattice, and mobile robots. This would be a nice future work section and extension on the paper.
% /////////////////////////////////////////////////////////////////////////////////////////////
% \cite{goossens93}

\end{document}
