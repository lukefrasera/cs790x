\documentclass{article}
\usepackage{hyperref}
\usepackage{natbib}
\usepackage[right=1in,top=1in,left=1in,bottom=1in]{geometry}

\begin{document}
\title{{\large Review} \\ PAPER_TITLE_HERE}
\author{Luke Fraser}
\date{\today}
\maketitle

% REFERENCE THE PAPER HERE ////////////////////////////////////////////////////////////////////
\begingroup
\renewcommand{\section}[2]{}
\bibliographystyle{plain}
\bibliography{references}
\endgroup

% /////////////////////////////////////////////////////////////////////////////////////////////
\section*{Summary}
% WRITE SUMMARY SECTION HERE //////////////////////////////////////////////////////////////////
In this paper the authors discuss their work on a smart object duplication method. The central idea is that you can use small smart blocks tht can link together to duplicat arbitrary object. It is basically another method of 3D printing. In the paper the authors use small 1 cm cubes that can link and communicate with each other. A distributed network is built that allows the blocks to talk with one another. With the ability to pass messages the blocks perform a distributed algorithm to map the object and link other blocks to reproduce the same shape out of the blocks. From a high-level context in theory the medium doesn't matter with the algorithm development in the paper. The use of arbitrary blocks of arbitrary size could be used to produce better results, however due to hardware limitations 1 cm cubes is the smallest achievable at the time of the paper. The hope is that as technology improves more blocks could be used to produce high-resolution representations of objects.

20 1 cm cubes were used in to copy several shapes in the paper. The blocks are first 
% /////////////////////////////////////////////////////////////////////////////////////////////
\section*{Strengths}
% DISCUSS THE STRENGTHS OF THE PAPER //////////////////////////////////////////////////////////

% /////////////////////////////////////////////////////////////////////////////////////////////
\section*{Critique}
% DISCUSS THE CRITIQUE OF THE PAPER ///////////////////////////////////////////////////////////

% /////////////////////////////////////////////////////////////////////////////////////////////
\cite{goossens93}

\end{document}
