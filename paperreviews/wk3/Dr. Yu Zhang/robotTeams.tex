\documentclass{article}
\usepackage{hyperref}
\usepackage{natbib}
\usepackage[right=1in,top=1in,left=1in,bottom=1in]{geometry}

\begin{document}
\title{{\large Review} \\ Feb. 6th Robot Teams: Dr. Yu Zhang}
\author{Luke Fraser}
\date{\today}
\maketitle

% REFERENCE THE PAPER HERE ////////////////////////////////////////////////////////////////////
\begingroup
\renewcommand{\section}[2]{}
\bibliographystyle{plain}
\bibliography{references}
\endgroup

% /////////////////////////////////////////////////////////////////////////////////////////////
\section*{Summary}
% WRITE SUMMARY SECTION HERE //////////////////////////////////////////////////////////////////
In the presentation Dr. Yu Zhang went over many different areas of his research. He discussed his work in robotic task allocation, robot team communication, path following, and much more.  His talk covered  a wide range of robotics topics dealing with robots working in teams. Dr. Yu Zhang addressed his research particularly in the area of robots with limited capabilities.

In one of his sections he discussed a method for producing a solution to a task problem when there are dependencies that require a collaboration between two robots. the main contribution of this work was to identify tasks that require a collaboration between several robots. He did not discuss the details of his method, but he did discuss how loops in networks form and these loops represent a situation that requires robotic collaboration. He tested his method on a robot scenario where a robot was tasked with retrieving a diamond from a room. When the robot touched the diamond a door would close that could only be opened from the outside. The task dependencies prevent a single robot from completing the task. His method was able to recognize this dependency and use two robots to complete the task.

Another section of his presentation discussed task allocation. Again he did not go into detail on the method for implementation of his task allocation. He discussed he was able to perform allocation by simplifying the problem into something that could be solved in polynomial time. He was able to allocate tasks fro robot teams in near real-time with up to roughly 6 members. These teams are fairly small groups, but the allocation of tasks allows them to work seamlessly together.
% /////////////////////////////////////////////////////////////////////////////////////////////
\section*{Strengths}
% DISCUSS THE STRENGTHS OF THE PAPER //////////////////////////////////////////////////////////
Dr. Yu Zhang showed that he has worked on a lot of the problems that plague multi-robot systems. He found solutions to the common problems and explained the contribution of each of his works. He also explained where his research is marketable for funding. This was useful information for understanding how well his research will be funded by other associations. Overall he addressed a lot of useful areas of his research.
% /////////////////////////////////////////////////////////////////////////////////////////////
\section*{Critique}
% DISCUSS THE CRITIQUE OF THE PAPER ///////////////////////////////////////////////////////////
Dr. Yu Zhang's talk was spread to thin with explanation of his methods. Coming in with little background knowledge of his research made it very difficult to understand how he implemented a lof of his research. I had to accept a lot of what he claimed because his presentation lacked a lot of background information. It seemed he was trying to go over everything he had ever done in his field rather than give an informative colloquia. It would have been more interesting if he had left out many section and just discussed a few of the topics in more detail. I left feeling like I didn't get much out of his talk.

Dr. Yu Zhang's presentation didn't feel very complete. There was no real theme or goal to the talk. I didn't really understand what he was trying to accomplish. Overall the presentation felt like a sales pitch. I understand he is applying for a job, but I don't think it should feel so rushed and lack luster.
% /////////////////////////////////////////////////////////////////////////////////////////////

\end{document}
