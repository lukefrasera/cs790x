\documentclass{article}
\usepackage{hyperref}
\usepackage{natbib}
\usepackage[right=1in,top=1in,left=1in,bottom=1in]{geometry}

\begin{document}
\title{{\large Review} \\ Chapter: 38 Behavior-Based Systems}
\author{Luke Fraser}
\date{\today}
\maketitle

% REFERENCE THE PAPER HERE ////////////////////////////////////////////////////////////////////
\begingroup
\renewcommand{\section}[2]{}
\bibliographystyle{plain}
\bibliography{references}
\endgroup

% /////////////////////////////////////////////////////////////////////////////////////////////
\section*{Summary}
% WRITE SUMMARY SECTION HERE //////////////////////////////////////////////////////////////////
In this chapter the authors describe behavior based systems and the benefits and limitations of such systems. Behavior based systems are implemented using a network of behaviors. These behavior-nodes can interact with other behaviors in the graph to produce actuator output. The difficulty in behavior-based systems is deciding which behaviors have control over the robot. There are many schemas that have been used to perform action selection. Some as simple as a heirarchical structure of behaviors and other that become more computationally intensive such as activation spreading, fuzzy logic, probability theory, etc \ldots. The goal of a behavior based approach is to simplify the actions of a robot into nodes of simple behaviors. This presents a smaller state space which is easier to develop.

An important aspect of behavior based systems is the use of learning. Learning allows control architectures to adapt to the environment or learn how to complete an arbitrary task. Examples of include learning to walk and mapping an environment. Behavior based approaches have taken advantage of reinforcement learning as well behavior networks to adapt in environments. As well performing action selection can be difficult and learning has helped improve action selection of behavior-based systems.
% /////////////////////////////////////////////////////////////////////////////////////////////
\section*{Strengths}
% DISCUSS THE STRENGTHS OF THE PAPER //////////////////////////////////////////////////////////
This chapter defined behavior based approached very well. It gave a great outline of how behavior based systems have been used as well as what the motivation for using such systems is. The benefits of behavior based approaches is clear as well as the issues.
% /////////////////////////////////////////////////////////////////////////////////////////////
% \section*{Critique}
% DISCUSS THE CRITIQUE OF THE PAPER ///////////////////////////////////////////////////////////

% /////////////////////////////////////////////////////////////////////////////////////////////
\cite{book}

\end{document}
