\documentclass{article}
\usepackage{hyperref}
\usepackage{natbib}
\usepackage[right=1in,top=1in,left=1in,bottom=1in]{geometry}

\begin{document}
\title{{\large Review} \\ Networked Robots}
\author{Luke Fraser}
\date{\today}
\maketitle

% REFERENCE THE PAPER HERE ////////////////////////////////////////////////////////////////////
\begingroup
\renewcommand{\section}[2]{}
\bibliographystyle{plain}
\bibliography{references}
\endgroup

% /////////////////////////////////////////////////////////////////////////////////////////////
\section*{Summary}
% WRITE SUMMARY SECTION HERE //////////////////////////////////////////////////////////////////
In this chapter the authors review networked robotics. The use of networks is a very important part of robotics. Networks are used in almost every aspect of robotics. It is how all communication is done across most complex robotics systems. Networks can be used as a mechanism for control as well as a means for communications between robotic systems. In the case of multi-robot systems networks are used to robots to communicate between each other. A robot that needs to pass information to another will do this over a network. usually networks consist of unique IP addresses assigned to each robot in the system. Communication to each robot is done through UDP/TCP messages usually. These are standard internet protocols of communication.

One of the difficulties of managing robotic systems with networks is knowing who should communicate with who and when as well as how much information to share. Another major challenge of robotic networks is that they are dynamic. They do not remain in a static form as robots are capable of moving throughout their environment. Communication in a dynamic network is difficult because robots don't have a consistent connection with each other. Distance becomes a factor especially when working with wireless networks. Wireless networks have a limited distance that they can maintain steady communication.

Another major issue of networked robotics is bandwidth. The more communication that each robots send to another will cause the network to clog with traffic and make it unable to communication to occur. Due to these issues different strategies have been developed to cope with the control issues with network robots. Two types of strategies have been developed distributed and centralized. Centralize solutions to multi-robot control was used in early systems. It is much easier to program a cntralized system of robots. A centralized systems consists of a single master control node that sends control messages to all agents in the system. This model for control  became difficult to manage and as more robots are added to the system communication floods the system. The main drawback of centralized systems is the single source of failure as well as the single source control.
% /////////////////////////////////////////////////////////////////////////////////////////////
\section*{Strengths}
% DISCUSS THE STRENGTHS OF THE PAPER //////////////////////////////////////////////////////////

% /////////////////////////////////////////////////////////////////////////////////////////////
\section*{Critique}
% DISCUSS THE CRITIQUE OF THE PAPER ///////////////////////////////////////////////////////////

% /////////////////////////////////////////////////////////////////////////////////////////////
\cite{ref}

\end{document}
