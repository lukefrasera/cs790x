\documentclass{article}
\usepackage{hyperref}
\usepackage{natbib}
\usepackage[right=1in,top=1in,left=1in,bottom=1in]{geometry}

\begin{document}
\title{{\large Review} \\ Safe Multirobot Navigation Within Dynamics Constraints}
\author{Luke Fraser}
\date{\today}
\maketitle

% REFERENCE THE PAPER HERE ////////////////////////////////////////////////////////// //////////
\begingroup
\renewcommand{\section}[2]{}
\bibliographystyle{plain}
\bibliography{references}
\endgroup

% /////////////////////////////////////////////////////////////////////////////////////////////
\section*{Summary}
% WRITE SUMMARY SECTION HERE //////////////////////////////////////////////////////////////////

% /////////////////////////////////////////////////////////////////////////////////////////////
\section*{Strengths}
% DISCUSS THE STRENGTHS OF THE PAPER //////////////////////////////////////////////////////////

% /////////////////////////////////////////////////////////////////////////////////////////////
\section*{Critique}
% DISCUSS THE CRITIQUE OF THE PAPER ///////////////////////////////////////////////////////////
This paper was written in a very scattered and unorganized fashion. It would have been nice if the paper could have been organized into smaller sections that were more specific. The paper also didn't show any mathematical expressions representing their methods or algorithm listings to help the reader quickly understand their methods. finding information of interest in this paper was not easy. In each section the prior work isn't separated from the authors methods which is difficult to locate within the text.
% /////////////////////////////////////////////////////////////////////////////////////////////
\cite{1677952}

\end{document}
